\title{TCCEC}

\documentclass[
	% -- opções da classe memoir --
	12pt,				% tamanho da fonte
	openright,			% capítulos começam em pág ímpar (insere página vazia caso preciso)
	oneside,			% para impressão em verso e anverso. Oposto a twoside
	a4paper,			% tamanho do papel. 
	% -- opções da classe abntex2 --
	chapter=TITLE,		% títulos de capítulos convertidos em letras maiúsculas
	section=TITLE,		% títulos de seções convertidos em letras maiúsculas
	%subsection=TITLE,	% títulos de subseções convertidos em letras maiúsculas
	%subsubsection=TITLE,% títulos de subsubseções convertidos em letras maiúsculas
	% -- opções do pacote babel --
	english,			% idioma adicional para hifenização
	french,				% idioma adicional para hifenização
	spanish,			% idioma adicional para hifenização
	brazil				% o último idioma é o principal do documento
	]{abntex2}

% ---
% Pacotes básicos 
% ---
\usepackage{lmodern}			% Usa a fonte Latin Modern
\usepackage[T1]{fontenc}		% Selecao de codigos de fonte.
\usepackage[utf8]{inputenc}		% Codificacao do documento (conversão automática dos acentos)
\usepackage{lastpage}			% Usado pela Ficha catalográfica
\usepackage{indentfirst}		% Indenta o primeiro parágrafo de cada seção.
\usepackage{color}				% Controle das cores
\usepackage{graphicx}			% Inclusão de gráficos
\usepackage{microtype} 			% para melhorias de justificação
% ---
		
% ---
% Pacotes adicionais, usados apenas no âmbito do Modelo Canônico do abnteX2
% ---
\usepackage{lipsum}				% para geração de dummy text
% ---

% ---
% Pacotes de citações
% ---
\usepackage[brazilian,hyperpageref]{backref}	 % Paginas com as citações na bibl
\usepackage[num,abnt-etal-list=0,abnt-etal-cite=3]{abntex2cite}	% Citações padrão ABNT

\usepackage{cite}
\renewcommand\citeleft{[}
\renewcommand\citeright{]}

% --- 
% CONFIGURAÇÕES DE PACOTES
% --- 

% ---
% Configurações do pacote backref
% Usado sem a opção hyperpageref de backref
\renewcommand{\backrefpagesname}{Citado na(s) página(s):~}
% Texto padrão antes do número das páginas
\renewcommand{\backref}{}
% Define os textos da citação
\renewcommand*{\backrefalt}[4]{
	\ifcase #1 %
		Nenhuma citação no texto.%
	\or
		Citado na página #2.%
	\else
		Citado #1 vezes nas páginas #2.%
	\fi}%
% ---

% ---
% Informações de dados para CAPA e FOLHA DE ROSTO
% ---
\titulo{Relatório de Estágio de Docência}
\autor{Carlos David Braga Borges}
\local{SOBRAL}
\data{2019}
\orientador{Jarbas Joaci de Mesquita Sá Junior}
\coorientador{}
\instituicao{%
  Universidade Federal do Ceará \par
  Curso de Engenharia de Computação \par
  Campus Sobral}
\tipotrabalho{Relatório}
% O preambulo deve conter o tipo do trabalho, o objetivo, 
% o nome da instituição e a área de concentração 
\preambulo{O estágio de docência do estudante Carlos David Braga Borges é adequado para a aprovação na disciplina de Estágio de Docência no Programa de Pós-Graduação em Engenharia Elétrica e de Computação (PPGEEC).}
% ---

% ---
% Configurações de aparência do PDF final

% alterando o aspecto da cor azul
\definecolor{blue}{RGB}{41,5,195}

% informações do PDF
\makeatletter
\hypersetup{
     	%pagebackref=true,
		pdftitle={\@title}, 
		pdfauthor={},
    	pdfsubject={\imprimirpreambulo},
	    pdfcreator={LaTeX with abnTeX2},
		pdfkeywords={wow}{much work}{so engineer}, 
		colorlinks=true,       		% false: boxed links; true: colored links
    	linkcolor=blue,          	% color of internal links
    	citecolor=blue,        		% color of links to bibliography
    	filecolor=magenta,      		% color of file links
		urlcolor=blue,
		bookmarksdepth=4
}
\makeatother
% --- 

% --- 
% Espaçamentos entre linhas e parágrafos 
% --- 

% O tamanho do parágrafo é dado por:
\setlength{\parindent}{1.3cm}

% Controle do espaçamento entre um parágrafo e outro:
\setlength{\parskip}{0.2cm}  % tente também \onelineskip

% ---
% compila o indice
% ---
\makeindex
% ---

\usepackage{silence}
\WarningFilter{caption}{Unsupported document class}
\WarningFilter{caption}{The option `hypcap=true' will be ignored}

% Pacotes extras utilizados no trabalho
\usepackage{amsmath}
\usepackage{amssymb}
\usepackage{nomencl}
\renewcommand{\nomname}{\listadesiglasname}
\makenomenclature
\newcommand{\citet}[1]{\citeauthor{#1} (\citeyear{#1})}
\usepackage{float}
\usepackage{enumitem}  
\SetLabelAlign{parright}{\strut\smash{\parbox[t]{\labelwidth}{\raggedleft#1}}}
\setlist[description]{style=multiline,leftmargin=1.2cm,noitemsep}
\usepackage{array}
\newcolumntype{P}[1]{>{\centering\arraybackslash}p{#1}}
\usepackage[caption=false]{subfig}
\usepackage{mathtools}
\usepackage{amsfonts}
\usepackage{enumitem}
\setlist[itemize]{nosep}
\setlist[enumerate]{nosep}
\usepackage{algpseudocode}
\usepackage[boxed]{algorithm}
\usepackage[table, dvipsnames]{xcolor}
\usepackage{multirow}
\usepackage{pbox}
\usepackage[export]{adjustbox}

\def\changemargin#1#2{\list{}{\rightmargin#2\leftmargin#1}\item[]}
\let\endchangemargin=\endlist
\usepackage{babel}
\usepackage[table]{xcolor}
\usepackage{collcell}
\usepackage{hhline}
\usepackage{pgf}
\usepackage{multirow}
\def\colorModel{hsb} %You can use rgb or hsb
\newcommand\ColCell[1]{
  \pgfmathparse{#1<150?1:0}  %Threshold for changing the font color into the cells
    \ifnum\pgfmathresult=0\relax\color{white}\fi
  \pgfmathsetmacro\compA{0}      %Component R or H
  \pgfmathsetmacro\compB{#1/15} %Component G or S
  \pgfmathsetmacro\compC{1}      %Component B or B
  \edef\x{\noexpand\centering\noexpand\cellcolor[\colorModel]{\compA,\compB,\compC}}\x #1
  } 
\newcolumntype{E}{>{\collectcell\ColCell}m{0.4cm}<{\endcollectcell}}  %Cell width
\newcommand*\rot{\rotatebox{90}}

\usepackage{caption}

% ----
% Início do documento
% ----
\begin{document}

% Retira espaço extra obsoleto entre as frases.
\frenchspacing 

% ----------------------------------------------------------
% ELEMENTOS PRÉ-TEXTUAIS
% ----------------------------------------------------------
\pretextual

% ---
% Capa
% ---
\begin{figure}
\centering
\includegraphics[scale=0.6]{graphics/ufc}
\end{figure}

\begin{center}
\ABNTEXchapterfont\large
\textbf{UNIVERSIDADE FEDERAL DO CEARÁ} \\
\textbf{CAMPUS SOBRAL} \\
\textbf{PROGRAMA DE PÓS-GRADUAÇÃO EM ENGENHARIA ELÉTRICA E DE COMPUTAÇÃO} \\
\vspace{1cm}
\ABNTEXchapterfont\large

\end{center}

\vspace{1cm}

\imprimircapa


\setlength\parindent{0pt}

\begin{folhadeaprovacao}
	
	\begin{center}
		{\ABNTEXchapterfont\large\imprimirautor}
		
		\vspace{-8cm}
		
		\vspace*{\fill}\vspace*{\fill}
		\begin{center}
			\ABNTEXchapterfont\bfseries\Large\imprimirtitulo
		\end{center}
		\vspace*{\fill}
		
		\hspace{.45\textwidth}
		\begin{minipage}{.5\textwidth}
			\imprimirpreambulo
		\end{minipage}%
		\vspace*{\fill}
	\end{center}
	
	\vspace{-1cm}
	
	\assinatura{\textbf{\imprimirautor} \\ Aluno}
	\assinatura{\textbf{\imprimirorientador} \\ Orientador} 
	
\end{folhadeaprovacao}

\tableofcontents

\pagestyle{headings}
\chapter{Identificação do aluno}

\textbf{Nome do Aluno:} Carlos David Braga Borges

\textbf{Matrícula:} 431976

\textbf{IES na qual o estágio foi realizado:} Universidade Federal do Ceará -- Campus Sobral

\textbf{Curso de graduação no qual o estágio foi realizado:} Engenharia de Computação

\textbf{Disciplina na qual o estágio foi realizado:} Inteligência Computacional

\textbf{Professor responsável pela disciplina}: Jarbas Joaci de Mesquita Sá Junior

\textbf{Semestre da disciplina}: 5º semestre

\textbf{Período:} 2019.1


\newpage

\chapter{Atividades relacionadas à regência de classe}

\newcommand{\tabitem}{~~\llap{\textbullet}~~}

\section{Aula sobre Sistemas Fuzzy}
\begin{tabular}{ |l|l|l|l| }
\hline
\multicolumn{4}{ |c| }{\cellcolor{gray!50} \textbf{Atividades realizadas}} \\ \hline 
Semestre: 6º & Disciplina: Inteligência Computacional & Aula 1 & Data: 22/03/2019 \\ \hline
\multicolumn{2}{ |c| }{Tema: Sistemas Fuzzy} & \multicolumn{2}{ c| }{Duração da aula: 2h} \\ \hline
\multicolumn{4}{ |c| }{\cellcolor{gray!50} \textbf{Objetivos}} \\ \hline 
\multicolumn{4}{ |c| }{
	\vtop{
		\hbox{\strut \tabitem Objetivo 1}
		\hbox{\strut \tabitem Objetivo 2}
		\hbox{\strut \tabitem Objetivo 3}
	}
} \\ \hline 
\multicolumn{4}{ |c| }{\cellcolor{gray!50} \textbf{Síntese do conteúdo}} \\ \hline 
\multicolumn{4}{ |c| }{
	!!!
} \\ \hline
\end{tabular}

\section{Aula sobre Perceptron Multicamadas}
\begin{tabular}{ |l|l|l|l| }
	\hline
	\multicolumn{4}{ |c| }{\cellcolor{gray!50} \textbf{Atividades realizadas}} \\ \hline 
	Semestre: 6º & Disciplina: Inteligência Computacional & Aula 1 & Data: 22/03/2019 \\ \hline
	\multicolumn{2}{ |c| }{Tema: Sistemas Fuzzy} & \multicolumn{2}{ c| }{Duração da aula: 2h} \\ \hline
	\multicolumn{4}{ |c| }{\cellcolor{gray!50} \textbf{Objetivos}} \\ \hline 
	\multicolumn{4}{ |c| }{
		\vtop{
			\hbox{\strut \tabitem Objetivo 1}
			\hbox{\strut \tabitem Objetivo 2}
			\hbox{\strut \tabitem Objetivo 3}
		}
	} \\ \hline 
	\multicolumn{4}{ |c| }{\cellcolor{gray!50} \textbf{Síntese do conteúdo}} \\ \hline 
	\multicolumn{4}{ |c| }{
		!!!
	} \\ \hline
\end{tabular}

\section{Aula sobre Algoritmos Genéticos}
\begin{tabular}{ |l|l|l|l| }
	\hline
	\multicolumn{4}{ |c| }{\cellcolor{gray!50} \textbf{Atividades realizadas}} \\ \hline 
	Semestre: 6º & Disciplina: Inteligência Computacional & Aula 1 & Data: 22/03/2019 \\ \hline
	\multicolumn{2}{ |c| }{Tema: Sistemas Fuzzy} & \multicolumn{2}{ c| }{Duração da aula: 2h} \\ \hline
	\multicolumn{4}{ |c| }{\cellcolor{gray!50} \textbf{Objetivos}} \\ \hline 
	\multicolumn{4}{ |c| }{
		\vtop{
			\hbox{\strut \tabitem Objetivo 1}
			\hbox{\strut \tabitem Objetivo 2}
			\hbox{\strut \tabitem Objetivo 3}
		}
	} \\ \hline 
	\multicolumn{4}{ |c| }{\cellcolor{gray!50} \textbf{Síntese do conteúdo}} \\ \hline 
	\multicolumn{4}{ |c| }{
		!!!
	} \\ \hline
\end{tabular}

\section{Primeira aula sobre Redes Neurais Convolucionais}
\begin{tabular}{ |l|l|l|l| }
	\hline
	\multicolumn{4}{ |c| }{\cellcolor{gray!50} \textbf{Atividades realizadas}} \\ \hline 
	Semestre: 6º & Disciplina: Inteligência Computacional & Aula 1 & Data: 22/03/2019 \\ \hline
	\multicolumn{2}{ |c| }{Tema: Sistemas Fuzzy} & \multicolumn{2}{ c| }{Duração da aula: 2h} \\ \hline
	\multicolumn{4}{ |c| }{\cellcolor{gray!50} \textbf{Objetivos}} \\ \hline 
	\multicolumn{4}{ |c| }{
		\vtop{
			\hbox{\strut \tabitem Objetivo 1}
			\hbox{\strut \tabitem Objetivo 2}
			\hbox{\strut \tabitem Objetivo 3}
		}
	} \\ \hline 
	\multicolumn{4}{ |c| }{\cellcolor{gray!50} \textbf{Síntese do conteúdo}} \\ \hline 
	\multicolumn{4}{ |c| }{
		!!!
	} \\ \hline
\end{tabular}

\section{Segunda aula sobre Redes Neurais Convolucionais}
\begin{tabular}{ |l|l|l|l| }
	\hline
	\multicolumn{4}{ |c| }{\cellcolor{gray!50} \textbf{Atividades realizadas}} \\ \hline 
	Semestre: 6º & Disciplina: Inteligência Computacional & Aula 1 & Data: 22/03/2019 \\ \hline
	\multicolumn{2}{ |c| }{Tema: Sistemas Fuzzy} & \multicolumn{2}{ c| }{Duração da aula: 2h} \\ \hline
	\multicolumn{4}{ |c| }{\cellcolor{gray!50} \textbf{Objetivos}} \\ \hline 
	\multicolumn{4}{ |c| }{
		\vtop{
			\hbox{\strut \tabitem Objetivo 1}
			\hbox{\strut \tabitem Objetivo 2}
			\hbox{\strut \tabitem Objetivo 3}
		}
	} \\ \hline 
	\multicolumn{4}{ |c| }{\cellcolor{gray!50} \textbf{Síntese do conteúdo}} \\ \hline 
	\multicolumn{4}{ |c| }{
		!!!
	} \\ \hline
\end{tabular}

\vspace{1cm}

\begin{tabular}{ |l|l|l|l| }
	\hline
	\multicolumn{4}{ |c| }{\cellcolor{gray!50} \textbf{Atividades realizadas}} \\ \hline 
	Semestre: 6º & Disciplina: Inteligência Computacional & Aula 1 & Data: 22/03/2019 \\ \hline
	\multicolumn{2}{ |c| }{Tema: Sistemas Fuzzy} & \multicolumn{2}{ c| }{Duração da aula: 2h} \\ \hline
	\multicolumn{4}{ |c| }{\cellcolor{gray!50} \textbf{Objetivos}} \\ \hline 
	\multicolumn{4}{ |c| }{
		\vtop{
			\hbox{\strut \tabitem Objetivo 1}
			\hbox{\strut \tabitem Objetivo 2}
			\hbox{\strut \tabitem Objetivo 3}
		}
	} \\ \hline 
	\multicolumn{4}{ |c| }{\cellcolor{gray!50} \textbf{Síntese do conteúdo}} \\ \hline 
	\multicolumn{4}{ |c| }{
		!!!
	} \\ \hline
\end{tabular}


\newpage

\chapter{Atividades de preparação das aulas}

\section{Carga horária utilizada para preparação das aulas}

\begin{center}
	\begin{tabular}{ |c|c|c|c| }
		\hline
		\cellcolor{gray!50} \textbf{Aula} & \cellcolor{gray!50} \textbf{Data} & 
		\cellcolor{gray!50} \vtop{
			\hbox{\strut \textbf{Carga Horária}}
			\hbox{\strut \textbf{(Sala de aula)}}
		} &
		\cellcolor{gray!50} \vtop{
			\hbox{\strut \ \ \ \ \textbf{Carga Horária}}
			\hbox{\strut \textbf{(Preparação de aula)}}
		} \\ \hline
		Aula 1 & 22/03/2019 & 2h & 10h \\ \hline
		Aula 2 & 10/05/2019 & 2h & 16h \\ \hline
		Aula 3 & 31/05/2019 & 2h & 16h \\ \hline
		Aula 4 & 05/06/2019 & 2h & 20h \\ \hline
		Aula 5 & 07/06/2019 & 2h & 20h \\ \hline
		Aula 6 & 14/06/2019 & 2h & 4h \\ \hline
		\cellcolor{gray!50} \textbf{Total} & \cellcolor{gray!50} & \cellcolor{gray!50} \textbf{12h} & \cellcolor{gray!50} \textbf{86h} \\ \hline
	\end{tabular}
\end{center}

\section{Bibliografia}

\section{Material didático produzido}


\newpage
\chapter{Autoavaliação sobre o estágio de docência}

\section{Autocrítica sobre o desempenho do mestrando}

\section{Sugestões para o aprimoramento da disciplina}

\section{Resultados da avaliação docente}

\section{Contribuição do estágio para a formação profisisonal do aluno de mestrado}


\newpage
\chapter{Parecer do professor responsável}

\postextual
\newpage
\pagestyle{empty}
% ----------------------------------------------------------

% ----------------------------------------------------------
% Referências bibliográficas
% ----------------------------------------------------------

%\bibliographystyle{abntex2-num}
%\bibliography{references}

\end{document}
