\title{TCCEC}

\documentclass[
	% -- opções da classe memoir --
	12pt,				% tamanho da fonte
	openright,			% capítulos começam em pág ímpar (insere página vazia caso preciso)
	oneside,			% para impressão em verso e anverso. Oposto a twoside
	a4paper,			% tamanho do papel. 
	% -- opções da classe abntex2 --
	chapter=TITLE,		% títulos de capítulos convertidos em letras maiúsculas
	section=TITLE,		% títulos de seções convertidos em letras maiúsculas
	%subsection=TITLE,	% títulos de subseções convertidos em letras maiúsculas
	%subsubsection=TITLE,% títulos de subsubseções convertidos em letras maiúsculas
	% -- opções do pacote babel --
	english,			% idioma adicional para hifenização
	french,				% idioma adicional para hifenização
	spanish,			% idioma adicional para hifenização
	brazil				% o último idioma é o principal do documento
	]{abntex2}

% ---
% Pacotes básicos 
% ---
\usepackage{lmodern}			% Usa a fonte Latin Modern
\usepackage[T1]{fontenc}		% Selecao de codigos de fonte.
\usepackage[utf8]{inputenc}		% Codificacao do documento (conversão automática dos acentos)
\usepackage{lastpage}			% Usado pela Ficha catalográfica
\usepackage{indentfirst}		% Indenta o primeiro parágrafo de cada seção.
\usepackage{color}				% Controle das cores
\usepackage{graphicx}			% Inclusão de gráficos
\usepackage{microtype} 			% para melhorias de justificação
% ---
		
% ---
% Pacotes adicionais, usados apenas no âmbito do Modelo Canônico do abnteX2
% ---
\usepackage{lipsum}				% para geração de dummy text
% ---

% ---
% Pacotes de citações
% ---
\usepackage[brazilian,hyperpageref]{backref}	 % Paginas com as citações na bibl
\usepackage[num,abnt-etal-list=0,abnt-etal-cite=3]{abntex2cite}	% Citações padrão ABNT

\usepackage{cite}
\renewcommand\citeleft{[}
\renewcommand\citeright{]}

% --- 
% CONFIGURAÇÕES DE PACOTES
% --- 

% ---
% Configurações do pacote backref
% Usado sem a opção hyperpageref de backref
\renewcommand{\backrefpagesname}{Citado na(s) página(s):~}
% Texto padrão antes do número das páginas
\renewcommand{\backref}{}
% Define os textos da citação
\renewcommand*{\backrefalt}[4]{
	\ifcase #1 %
		Nenhuma citação no texto.%
	\or
		Citado na página #2.%
	\else
		Citado #1 vezes nas páginas #2.%
	\fi}%
% ---

% ---
% Informações de dados para CAPA e FOLHA DE ROSTO
% ---
\titulo{Relatório Final do Estágio de Docência}
\autor{Carlos David Braga Borges}
\local{SOBRAL}
\data{2019}
\orientador{Jarbas Joaci de Mesquita Sá Junior}
\coorientador{}
\instituicao{%
  Universidade Federal do Ceará \par
  Curso de Engenharia de Computação \par
  Campus Sobral}
\tipotrabalho{Relatório}
% O preambulo deve conter o tipo do trabalho, o objetivo, 
% o nome da instituição e a área de concentração 
\preambulo{O estágio realizado pelo aluno Carlos David Braga Borges é adequado para a aprovação na disciplina de Estágio de Docência no Programa de Pós-Graduação em Engenharia Elétrica e de Computação (PPGEEC).}
% ---

% ---
% Configurações de aparência do PDF final

% alterando o aspecto da cor azul
\definecolor{blue}{RGB}{41,5,195}

% informações do PDF
\makeatletter
\hypersetup{
     	%pagebackref=true,
		pdftitle={\@title}, 
		pdfauthor={},
    	pdfsubject={\imprimirpreambulo},
	    pdfcreator={LaTeX with abnTeX2},
		pdfkeywords={wow}{much work}{so engineer}, 
		colorlinks=true,       		% false: boxed links; true: colored links
    	linkcolor=blue,          	% color of internal links
    	citecolor=blue,        		% color of links to bibliography
    	filecolor=magenta,      		% color of file links
		urlcolor=blue,
		bookmarksdepth=4
}
\makeatother
% --- 

% --- 
% Espaçamentos entre linhas e parágrafos 
% --- 

% O tamanho do parágrafo é dado por:
\setlength{\parindent}{1.3cm}

% Controle do espaçamento entre um parágrafo e outro:
\setlength{\parskip}{0.2cm}  % tente também \onelineskip

% ---
% compila o indice
% ---
\makeindex
% ---

\usepackage{silence}
\WarningFilter{caption}{Unsupported document class}
\WarningFilter{caption}{The option `hypcap=true' will be ignored}

% Pacotes extras utilizados no trabalho
\usepackage{amsmath}
\usepackage{amssymb}
\let\newfloat\undefined
\usepackage{nomencl}
\renewcommand{\nomname}{\listadesiglasname}
\makenomenclature
\newcommand{\citet}[1]{\citeauthor{#1} (\citeyear{#1})}
\usepackage{float}
\usepackage{enumitem}  
\SetLabelAlign{parright}{\strut\smash{\parbox[t]{\labelwidth}{\raggedleft#1}}}
\setlist[description]{style=multiline,leftmargin=1.2cm,noitemsep}
\usepackage{array}
\newcolumntype{P}[1]{>{\centering\arraybackslash}p{#1}}
\usepackage[caption=false]{subfig}
\usepackage{mathtools}
\usepackage{amsfonts}
\usepackage{enumitem}
\setlist[itemize]{nosep}
\setlist[enumerate]{nosep}
\usepackage{algpseudocode}
\usepackage[boxed]{algorithm}
\usepackage[table, dvipsnames]{xcolor}
\usepackage{multirow}
\usepackage{pbox}
\usepackage{ragged2e}
\usepackage{forloop}
\usepackage[export]{adjustbox}

\def\changemargin#1#2{\list{}{\rightmargin#2\leftmargin#1}\item[]}
\let\endchangemargin=\endlist
\usepackage{babel}
\usepackage[table]{xcolor}
\usepackage{collcell}
\usepackage{hhline}
\usepackage{pgf}
\usepackage{multirow}
\def\colorModel{hsb} %You can use rgb or hsb
\newcommand\ColCell[1]{
  \pgfmathparse{#1<150?1:0}  %Threshold for changing the font color into the cells
    \ifnum\pgfmathresult=0\relax\color{white}\fi
  \pgfmathsetmacro\compA{0}      %Component R or H
  \pgfmathsetmacro\compB{#1/15} %Component G or S
  \pgfmathsetmacro\compC{1}      %Component B or B
  \edef\x{\noexpand\centering\noexpand\cellcolor[\colorModel]{\compA,\compB,\compC}}\x #1
  } 
\newcolumntype{E}{>{\collectcell\ColCell}m{0.4cm}<{\endcollectcell}}  %Cell width
\newcommand*\rot{\rotatebox{90}}

\usepackage{caption}

% ----
% Início do documento
% ----
\begin{document}

% Retira espaço extra obsoleto entre as frases.
\frenchspacing 

% ----------------------------------------------------------
% ELEMENTOS PRÉ-TEXTUAIS
% ----------------------------------------------------------
\pretextual

% ---
% Capa
% ---
\begin{figure}
\centering
\includegraphics[scale=0.6]{graphics/ufc}
\end{figure}

\begin{center}
\ABNTEXchapterfont\large
\textbf{UNIVERSIDADE FEDERAL DO CEARÁ -- CAMPUS SOBRAL} \\
\textbf{PROGRAMA DE PÓS-GRADUAÇÃO EM ENGENHARIA ELÉTRICA E DE COMPUTAÇÃO} \\
\vspace{1cm}
\ABNTEXchapterfont\large

\end{center}

\vspace{1cm}

\imprimircapa


\setlength\parindent{0pt}

\begin{folhadeaprovacao}
	
	\begin{center}
		{\ABNTEXchapterfont\large\imprimirautor}
		
		\vspace{-8cm}
		
		\vspace*{\fill}\vspace*{\fill}
		\begin{center}
			\ABNTEXchapterfont\bfseries\Large\imprimirtitulo
		\end{center}
		\vspace*{\fill}
		
		\hspace{.45\textwidth}
		\begin{minipage}{.5\textwidth}
			\imprimirpreambulo
		\end{minipage}%
		\vspace*{\fill}
	\end{center}
	
	\vspace{-1cm}
	
	\assinatura{\textbf{\imprimirautor} \\ Aluno}
	\assinatura{\textbf{\imprimirorientador} \\ Orientador} 
	
\end{folhadeaprovacao}

\tableofcontents

\pagestyle{plain}

\chapter{Identificação do aluno}

\textbf{Nome do Aluno:} Carlos David Braga Borges

\textbf{Matrícula:} 431976

\textbf{IES na qual o estágio foi realizado:} Universidade Federal do Ceará -- Campus Sobral

\textbf{Curso de graduação no qual o estágio foi realizado:} Engenharia de Computação

\textbf{Disciplina na qual o estágio foi realizado:} Inteligência Computacional

\textbf{Professor responsável pela disciplina}: Jarbas Joaci de Mesquita Sá Junior

\textbf{Semestre da disciplina}: 6º semestre

\textbf{Período:} 2019.1


\newpage

\chapter{Atividades relacionadas à regência de classe}

\vspace{-1cm}

\section{Informações sobre as aulas ministradas}

\newcommand{\tabitem}{~~\llap{\textbullet}~~}

\subsection{Aula sobre Sistemas Fuzzy}
\begin{tabular}{ |l|l|l|l| }
\hline
\multicolumn{4}{ |c| }{\cellcolor{gray!50} \textbf{Atividades realizadas}} \\ \hline 
Semestre: 6º & Disciplina: Inteligência Computacional & Aula 1 & Data: 22/03/2019 \\ \hline
\multicolumn{2}{ |c| }{Tema: Sistemas Fuzzy} & \multicolumn{2}{ c| }{Duração da aula: 2h} \\ \hline
\multicolumn{4}{ |c| }{\cellcolor{gray!50} \textbf{Objetivos}} \\ \hline 
\multicolumn{4}{ |c| }{
	\vtop{
		\hbox{\strut \tabitem Revisar os conceitos de lógica fuzzy;}
		\hbox{\strut \tabitem Conhecer aplicações de sistemas fuzzy, suas vantagens e desvantagens;}
		\hbox{\strut \tabitem Compreender as etapas de elaboração de um sistema fuzzy;}
	}
} \\ \hline 
\multicolumn{4}{ |c| }{\cellcolor{gray!50} \textbf{Síntese do conteúdo}} \\ \hline 
\multicolumn{4}{ |c| }{
	\vtop{
		\hbox{\strut \tabitem Conceitos de lógica fuzzy: conjuntos fuzzy e funções de pertinência;}
		\hbox{\strut \tabitem Comparação entre lógica fuzzy e lógica tradicional;}
		\hbox{\strut \tabitem Sistemas fuzzy: vantagens e desvantagens;}
		\hbox{\strut \tabitem Exemplos de uso de sistemas fuzzy na indústria;}
		\hbox{\strut \tabitem Processo de inferência fuzzy: determinação de entradas e saídas, conjuntos}
		\hbox{\strut fuzzy e funções de pertinência, regras fuzzy, agregação, defuzzificação;}
		\hbox{\strut \tabitem Exemplo prático de sistema fuzzy aplicado a controle de altitude de aeronaves.}
	}
} \\ \hline
\end{tabular}

\subsection{Aula sobre Perceptron Multicamadas}
\begin{tabular}{ |l|l|l|l| }
	\hline
	\multicolumn{4}{ |c| }{\cellcolor{gray!50} \textbf{Atividades realizadas}} \\ \hline 
	Semestre: 6º & Disciplina: Inteligência Computacional & Aula 2 & Data: 10/05/2019 \\ \hline
	\multicolumn{2}{ |c| }{Tema: Perceptron Multicamadas} & \multicolumn{2}{ c| }{Duração da aula: 2h} \\ \hline
	\multicolumn{4}{ |c| }{\cellcolor{gray!50} \textbf{Objetivos}} \\ \hline 
	\multicolumn{4}{ |c| }{
		\vtop{
			\hbox{\strut \tabitem Revisar os conceitos de redes neurais;}
			\hbox{\strut \tabitem Compreender os hiperparâmetros das redes neurais;}
			\hbox{\strut \tabitem Conhecer as vantagens e desvantagens de modelos preditivos neurais;}
			\hbox{\strut \tabitem Entender como avaliar um modelo neural;}
		}
	} \\ \hline 
	\multicolumn{4}{ |c| }{\cellcolor{gray!50} \textbf{Síntese do conteúdo}} \\ \hline 
	\multicolumn{4}{ |c| }{
		\vtop{
			\hbox{\strut \tabitem Conceitos de redes neurais: neurônio, sinapses, camadas, perda, otimização;}
			\hbox{\strut \tabitem Vantagens e desvantagens de modelos neurais preditivos;}
			\hbox{\strut \tabitem Hiperparâmetros de redes neurais: número de camadas, número de neurônios, }
			\hbox{\strut função de ativação, pesos, algoritmo de otimização, função de perda, critério de }
			\hbox{\strut parada, número de épocas de treinamento;}
			\hbox{\strut \tabitem Processos de validação: Holdout, K-Fold, Leave-one-out;}
			\hbox{\strut \tabitem Exemplo prático de rede neural MLP aplicada à segmentação de pele.}
		}
	} \\ \hline
\end{tabular}

\subsection{Aula sobre Algoritmos Genéticos}
\begin{tabular}{ |l|l|l|l| }
	\hline
	\multicolumn{4}{ |c| }{\cellcolor{gray!50} \textbf{Atividades realizadas}} \\ \hline 
	Semestre: 6º & Disciplina: Inteligência Computacional & Aula 3 & Data: 31/05/2019 \\ \hline
	\multicolumn{2}{ |c| }{Tema: Algoritmos Genéticos} & \multicolumn{2}{ c| }{Duração da aula: 2h} \\ \hline
	\multicolumn{4}{ |c| }{\cellcolor{gray!50} \textbf{Objetivos}} \\ \hline 
	\multicolumn{4}{ |c| }{
		\vtop{
			\hbox{\strut \tabitem Revisar os conceitos de algoritmos genéticos;}
			\hbox{\strut \tabitem Compreender a estrutura de algoritmos genéticos;}
			\hbox{\strut \tabitem Entender os possíveis parâmetros para otimização com algoritmos genéticos.}
		}
	} \\ \hline 
	\multicolumn{4}{ |c| }{\cellcolor{gray!50} \textbf{Síntese do conteúdo}} \\ \hline 
	\multicolumn{4}{ |c| }{
		\vtop{
			\hbox{\strut \tabitem Conceitos fundamentais: indivíduo, população, aptidão;}
			\hbox{\strut \tabitem Estrutura base de um algoritmo genético;}
			\hbox{\strut \tabitem Parâmetros importantes: representação do problema, inicialização da população,}
			\hbox{\strut função de aptidão, métodos de seleção, cruzamento e mutação, critérios de parada,}
			\hbox{\strut elitismo e métodos adaptativos;}
			\hbox{\strut \tabitem Vantagens e desvantagens de otimização por algoritmos genéticos;}
			\hbox{\strut \tabitem Exemplo prático aplicado ao problema do caixeiro viajante.}
		}
	} \\ \hline
\end{tabular}

\subsection{Primeira aula sobre Redes Neurais Convolucionais}
\begin{tabular}{ |l|l|l|l| }
	\hline
	\multicolumn{4}{ |c| }{\cellcolor{gray!50} \textbf{Atividades realizadas}} \\ \hline 
	Semestre: 6º & Disciplina: Inteligência Computacional & Aula 4 & Data: 05/06/2019 \\ \hline
	\multicolumn{2}{ |c| }{Tema: Redes Neurais Convolucionais} & \multicolumn{2}{ c| }{Duração da aula: 2h} \\ \hline
	\multicolumn{4}{ |c| }{\cellcolor{gray!50} \textbf{Objetivos}} \\ \hline 
	\multicolumn{4}{ |c| }{
		\vtop{
			\hbox{\strut \tabitem Conhecer os conceitos de convolução 1D e 2D;}
			\hbox{\strut \tabitem Entender o funcionamento das camadas de convolução e pooling;}
			\hbox{\strut \tabitem Compreender a arquitetura de redes neurais convolucionais.}
		}
	} \\ \hline 
	\multicolumn{4}{ |c| }{\cellcolor{gray!50} \textbf{Síntese do conteúdo}} \\ \hline 
	\multicolumn{4}{ |c| }{
		\vtop{
			\hbox{\strut \tabitem Aplicações de redes neurais convolucionais;}
			\hbox{\strut \tabitem Convolução 1D e 2D: filtros e operações;}
			\hbox{\strut \tabitem Camadas de convolução: filtros, funções e mapas de ativação, largura do passo;}
			\hbox{\strut \tabitem Camadas de pooling: métodos de agregação, tamanho do filtro, largura do passo;}
			\hbox{\strut \tabitem Arquitetura de uma rede neural convolucional;}
			\hbox{\strut \tabitem Pacotes para implementação de redes neurais convolucionais.}
		}
	} \\ \hline
\end{tabular}

\subsection{Segunda aula sobre Redes Neurais Convolucionais}
\begin{tabular}{ |l|l|l|l| }
	\hline
	\multicolumn{4}{ |c| }{\cellcolor{gray!50} \textbf{Atividades realizadas}} \\ \hline 
	Semestre: 6º & Disciplina: Inteligência Computacional & Aula 5 & Data: 07/06/2019 \\ \hline
	\multicolumn{2}{ |c| }{Tema: Redes Neurais Convolucionais} & \multicolumn{2}{ c| }{Duração da aula: 2h} \\ \hline
	\multicolumn{4}{ |c| }{\cellcolor{gray!50} \textbf{Objetivos}} \\ \hline 
	\multicolumn{4}{ |c| }{
		\vtop{
			\hbox{\strut \tabitem Revisar os conceitos de redes neurais convolucionais;}
			\hbox{\strut \tabitem Entender suas vantagens na aplicação à problemas com imagens;}
			\hbox{\strut \tabitem Conhecer possíveis otimizações para modelos convolucionais;}
			\hbox{\strut \tabitem Aprender a construir redes neurais convolucionais.}
		}
	} \\ \hline 
	\multicolumn{4}{ |c| }{\cellcolor{gray!50} \textbf{Síntese do conteúdo}} \\ \hline 
	\multicolumn{4}{ |c| }{
		\vtop{
			\hbox{\strut \tabitem Revisão de redes neurais convolucionais;}
			\hbox{\strut \tabitem Comparação com uma rede MLP para solução de problemas com imagens;}
			\hbox{\strut \tabitem Apresentação do dataset CIFAR-10;}
			\hbox{\strut \tabitem Exemplo prático de classificação usando MLP;}
			\hbox{\strut \tabitem Exemplo prático de classificação com rede neural convolucional;}
			\hbox{\strut \tabitem Uso de GPUs para treinamento de redes neurais;}
			\hbox{\strut \tabitem Possíveis otimizações: normalização, aumento de dados, taxa de aprendizado;}
			\hbox{\strut adaptativa, regularizações L1 e L2, dropout, normalização de lotes;}
			\hbox{\strut \tabitem Exemplo prático de classificação com rede neural convolucional otimizada.}
		}
	} \\ \hline
\end{tabular}

\subsection{Aula sobre o Trabalho Extra}
\begin{tabular}{ |l|l|l|l| }
	\hline
	\multicolumn{4}{ |c| }{\cellcolor{gray!50} \textbf{Atividades realizadas}} \\ \hline 
	Semestre: 6º & Disciplina: Inteligência Computacional & Aula 6 & Data: 14/06/2019 \\ \hline
	\multicolumn{2}{ |c| }{Tema: Trabalho Extra} & \multicolumn{2}{ c| }{Duração da aula: 2h} \\ \hline
	\multicolumn{4}{ |c| }{\cellcolor{gray!50} \textbf{Objetivos}} \\ \hline 
	\multicolumn{4}{ |c| }{
		\vtop{
			\hbox{\strut \tabitem Compreender os problemas do Trabalho Extra;}
			\hbox{\strut \tabitem Tirar dúvidas sobre os problemas e obter sugestões.}
		}
	} \\ \hline 
	\multicolumn{4}{ |c| }{\cellcolor{gray!50} \textbf{Síntese do conteúdo}} \\ \hline 
	\multicolumn{4}{ |c| }{
		\vtop{
			\hbox{\strut \tabitem Problema 1: aprimoramento do sistema fuzzy desenvolvido em aula;}
			\hbox{\strut \tabitem Problema 2: aplicação da rede neural desenvolvida em aula para classificação}
			\hbox{\strut de doenças dermatológicas;}
			\hbox{\strut \tabitem Problema 3: aplicação de do algoritmo genético apresentado em aula na solução}
			\hbox{\strut de uma variação do problema da mochila inteira.}
		}
	} \\ \hline
\end{tabular}

\section{Metodologia de ensino}

As aulas ministradas possuíram componentes teóricos e práticos.
Os componentes teóricos envolveram revisões de conteúdo, apresentação de novas ideias e otimizações, exemplos de uso na pesquisa e na indústria, vantagens e desvantagens e guias para implementação das técnicas mencionadas.
Para fortalecer o aprendizado da teoria, cada aula também teve um exemplo prático de aplicação, com explanações acerca da metodologia de implementação e apresentação de programas e códigos fonte.
Essa medida teve o objetivo de aproximar os estudantes do aspecto prático da engenharia de programas de inteligência computacional, envolvendo seus desafios e soluções.

\section{Metodologia de avaliação}

A avaliação final foi feita através de um trabalho extra com valor de 1,0 ponto na média final da disciplina de Inteligência Computacional.
O trabalho continha três questões diretamente associadas aos temas e códigos desenvolvidos em sala.
As questões desafiavam os estudantes a melhorar os códigos apresentados, com base no aprendizado obtido durante as aulas.
O trabalho passado aos estudantes pode ser visualizado no Anexo A.

\section{Resultado das avaliações}

O resultado não pôde ser estabelecido, pois ninguém enviou o trabalho extra.
A principal hipótese para o ocorrido é a sobrecarga dos estudantes no término do semestre e o fato de o trabalho ter sido passado como opcional.
Assim sendo, os estudantes, pressionados pelas demais disciplinas, optaram por não fazer o trabalho extra.

\newpage

\chapter{Atividades de preparação das aulas}

\vspace{-1cm}

\section{Carga horária utilizada para preparação das aulas}

A carga horária utilizada para preparação das aulas pode ser visualizada abaixo.

\begin{center}
	\begin{tabular}{ |c|c|c|c| }
		\hline
		\cellcolor{gray!50} \textbf{Aula} & \cellcolor{gray!50} \textbf{Data} & 
		\cellcolor{gray!50} \vtop{
			\hbox{\strut \textbf{Carga Horária}}
			\hbox{\strut \textbf{(Sala de aula)}}
		} &
		\cellcolor{gray!50} \vtop{
			\hbox{\strut \ \ \ \ \textbf{Carga Horária}}
			\hbox{\strut \textbf{(Preparação de aula)}}
		} \\ \hline
		Aula 1 & 22/03/2019 & 2h & 10h \\ \hline
		Aula 2 & 10/05/2019 & 2h & 16h \\ \hline
		Aula 3 & 31/05/2019 & 2h & 16h \\ \hline
		Aula 4 & 05/06/2019 & 2h & 20h \\ \hline
		Aula 5 & 07/06/2019 & 2h & 20h \\ \hline
		Aula 6 & 14/06/2019 & 2h & 4h \\ \hline
		\cellcolor{gray!50} \textbf{Total} & \cellcolor{gray!50} & \cellcolor{gray!50} \textbf{12h} & \cellcolor{gray!50} \textbf{86h} \\ \hline
	\end{tabular}
\end{center}

Como visto na tabela, o tempo de preparação de aula superou amplamente o tempo utilizado em sala.
Isso ocorreu, pois a preparação da aula incluiu a revisão do tema, a preparação de slides e elaboração de códigos fonte para os exemplos práticos.

\section{Bibliografia}

Os dois principais livros utilizados na elaboração das aulas são citados a seguir.

[1] Kahraman, C. \& Kaymak, U. \& Yazici, A. (2016). Fuzzy Logic in Its 50th Year. DOI: 10.1007/978-3-319-31093-0.

[2] Russell, S. J. \& Norvig, P., \& Davis, E. (2010). Artificial intelligence: a modern approach. 3rd ed. Upper Saddle River, NJ: Prentice Hall.

Diversos sites e blogs dedicados ao ensino de inteligência computacional foram estudados para a preparação das aulas. Apesar de não serem citados aqui, principalmente por conta da quantidade, esses materiais são citados nos slides de aula.

\newpage

\section{Material didático produzido}

Os materiais produzidos para as aulas foram slides e códigos de demonstração das técnicas.
Todas as aulas foram apresentadas com slides, exceto a primeira.
Para todas as aulas foram produzidos códigos de demonstração, com exceção da quarta aula, que foi teórica.
A tabela a seguir mostra a quantidade de páginas dos slides e linhas de código produzidas para cada aula.


\begin{center}
	\begin{tabular}{ |c|c|c|c| }
		\hline
		\cellcolor{gray!50} \textbf{Aula} & \cellcolor{gray!50} \textbf{Tema} &
		\cellcolor{gray!50} \vtop{
			\hbox{\strut \ \ \ \ \ \ \ \ \textbf{Slides}}
			\hbox{\strut \textbf{(Nº de Páginas)}}
		} &
		\cellcolor{gray!50} \vtop{
			\hbox{\strut \ \ \ \ \ \ \textbf{Código}}
			\hbox{\strut \textbf{(Nº de Linhas)}}
		} \\ \hline
		Aula 1 & Sistemas Fuzzy & 0 & 207 \\ \hline
		Aula 2 & Perceptron Multicamadas & 25 & 360 \\ \hline
		Aula 3 & Algoritmos Genéticos & 26 & 221 \\ \hline
		Aula 4 & Redes Neurais Convolucionais & 73 & 0 \\ \hline
		Aula 5 & Redes Neurais Convolucionais & 12 & 602 \\ \hline
		Aula 6 & Trabalho Extra &1 & 523 \\ \hline
		\cellcolor{gray!50} \textbf{Total} & \cellcolor{gray!50} & \cellcolor{gray!50} \textbf{137} & \cellcolor{gray!50} \textbf{1903} \\ \hline
	\end{tabular}
\end{center}

\newpage
\chapter{Autoavaliação sobre o estágio de docência}

\vspace{-1cm}

\section{Autocrítica sobre o desempenho do mestrando}

Ministrar aulas para uma turma de graduação foi uma experiência nova.
No começo, tentei fazer alguns experimentos, como não utilizar slides e usar bastante o quadro para ilustrar as ideias.
No entanto, esse método de aula não foi bem recebido pelos estudantes.
Depois disso, optei pelo método tradicional de aula com slides.
Acredito que trazer códigos e exemplos práticos em todas as aulas tenha sido um ponto forte.
Infelizmente, o trabalho final não foi entregue por nenhum dos estudantes.
Acredito que isso tenha ocorrido por conta da sobrecarga de término de semestre.
Provavelmente, se as questões tivessem sido passadas gradualmente durante o semestre, após cada aula, o trabalho teria sido realizado por completo pela maioria dos estudantes.
Apesar disso, acredito que os estudantes gostaram das aulas e conseguiram absorver bastante do conteúdo.

\section{Contribuição do estágio para a formação profisisonal do mestrando}

O estágio de docência contribuiu bastante para minha formação.
Foi muito interessante e esclarecedor ter a experiência de organizar os slides e códigos com o propósito de ensinar.
Também foi relevante observar o interesse dos estudantes para com o temas apresentados, principalmente durante a apresentação dos exemplos práticos.

\section{Resultados da avaliação docente}

Foram feitas três avaliações, nas aulas dos dias 22/03/19 (aula 1), 10/05/19 (aula 3) e 05/06/19 (aula 5). As respostas dos estudantes são mostradas nos gráficos a seguir.

\begin{figure}
	\centering
	\includegraphics[width=0.7\textwidth]{graphics/q1.png}
\end{figure}

\begin{figure}
	\centering
	\includegraphics[width=0.7\textwidth]{graphics/q2.png}
\end{figure}

\begin{figure}
	\centering
	\includegraphics[width=0.7\textwidth]{graphics/q3.png}
\end{figure}

\begin{figure}
	\centering
	\includegraphics[width=0.7\textwidth]{graphics/q4.png}
\end{figure}

\begin{figure}
	\centering
	\includegraphics[width=0.7\textwidth]{graphics/q5.png}
\end{figure}

\begin{figure}
	\centering
	\includegraphics[width=0.7\textwidth]{graphics/q6.png}
\end{figure}

\begin{figure}
	\centering
	\includegraphics[width=0.7\textwidth]{graphics/q7.png}
\end{figure}

\begin{figure}
	\centering
	\includegraphics[width=0.7\textwidth]{graphics/q8.png}
\end{figure}

Alguns estudantes responderam aos questionamentos opcionais da avaliação. Essas respostas são mostradas a seguir.

\textbf{Aula 1 -- 22/03/19}

(1) \textit{\textbf{Diga quais os pontos que mais lhe agradaram com relação à disciplina e ao professor:}}

$\vartriangleright$ O exemplo mostrado para ilustrar o conteúdo.

$\vartriangleright$ Professor didático, assunto bem organizado e em ordem lógica. Disciplina importantíssima p/ desenvolvimento de tecnologias futuras.

$\vartriangleright$ O professor falou de forma clara e simples, facilitando a compreensão do conteúdo.

$\vartriangleright$ Ter mostrado na prática a implementação dos códigos e explicado os mesmos na linguagem python.

$\vartriangleright$ Os exemplos da utilização na prática da Lógica Fuzzy em empresas e até mesmo no cotidiano foi bem interessante.

$\vartriangleright$ A apresentação de uma aplicação.

$\vartriangleright$ Boa didática com exemplos práticos.

$\vartriangleright$ Material usado em aula.

$\vartriangleright$ Ritmo de aula agradável. Boa escolha de ferramentas: Linux + Python.

$\vartriangleright$ Utilizou uma aplicação bem comentada com um exemplo simples da manutenção da altitude do drone. Espero que disponibilize o código e suas anotações, pois ofereceu vários exemplos na vida real.

$\vartriangleright$ Didático, domínio.

(2) \textit{\textbf{Diga quais os pontos que mais lhe desagradaram com relação à disciplina e ao professor:}}

$\vartriangleright$ Falta de slides.

$\vartriangleright$ Uso confuso do quadro branco.

$\vartriangleright$ Percebi que o docente não treinou sua aula, mas ao menos fez um plano de aula. Ficava esporadicamente usando as anotações do caderno para lembrar.

$\vartriangleright$ Sem opções.

(3) \textit{\textbf{Quais sugestões você daria para melhorar a disciplina:}}

$\vartriangleright$ Detalhar mais o conteúdo que está sendo abordado. Trazer as aplicações do assunto abordado em um slide.

$\vartriangleright$ Que continue.

$\vartriangleright$ Preparar slide para auxiliar na explicação.

$\vartriangleright$ Talvez o uso de slides melhorariam a dinâmica da aula, além de dar uma sequência dos passos a serem mostrados.

$\vartriangleright$ Utilizar um pouco mais o quadro branco para ilustrar conceitos abstratos enquanto está explicando.

$\vartriangleright$ Usar slides para ilustrar sua aula, treinar mais antes, acredito que o uso de animações como gifs ou uma sequência de slides seria mais didático. Lembre-se, quando os alunos fazem perguntas durante a aula sobre um conteúdo novo apresentado é que estão entendendo a aula.

$\vartriangleright$ Cobrança de maiores conhecimentos sobre programação.


\vspace{1cm}

\textbf{Aula 2 -- 10/05/19}

(1) \textit{\textbf{Diga quais os pontos que mais lhe agradaram com relação à disciplina e ao professor:}}

$\vartriangleright$ Os exemplos mostrados em sala.

$\vartriangleright$ Pragmatismo e cuidadoso com material didático,

$\vartriangleright$ Em relação à disciplina: assunto interessante, importante p/ futuras tecnologias. Em relação ao professor: forma de abordar o conteúdo.

$\vartriangleright$ Fala bem, conhece o conteúdo, utiliza bons exemplos na aula facilitando o entendimento.

$\vartriangleright$ Didática, material de ensino.

$\vartriangleright$ Trazer a implementação da teoria. Atenção em procurar trazer a teoria de forma visual.

$\vartriangleright$ O professor não usou nenhuma cola para auxiliar na aula, demonstrando seu domínio sobre o assunto. Código bem comentado.

(2) \textit{\textbf{Diga quais os pontos que mais lhe desagradaram com relação à disciplina e ao professor:}}

$\vartriangleright$ Voz baixa.

(3) \textit{\textbf{Quais sugestões você daria para melhorar a disciplina:}}

$\vartriangleright$ Um espaço maior para perguntas.

$\vartriangleright$ Os trabalhos poderiam ser passados assim que o conteúdo referente ao trabalho fosse ministrado.


\vspace{1cm}

\textbf{Aula 3 -- 05/06/19}

(1) \textit{\textbf{Diga quais os pontos que mais lhe agradaram com relação à disciplina e ao professor:}}

$\vartriangleright$ Conteúdo e didática.

$\vartriangleright$ Aplicação prática e fundamentos fortes matemáticos.

$\vartriangleright$ Quanto à disciplina, agrada-me o fato de abordar temas atuais e campo de estudo vasto. Quanto ao professor, agrada-me o fato da segurança e facilidade em transmitir o conteúdo bem como mostrar na prática o conteúdo mostrado.

(2) \textit{\textbf{Diga quais os pontos que mais lhe desagradaram com relação à disciplina e ao professor:}}

$\vartriangleright$ Nenhum.

$\vartriangleright$ Quanto à disciplina, alguns momentos foram desagradáveis por conta de certa dificuldade em entender alguns termos ligados ao conteúdo. Quanto ao professor, nada a declarar.

(3) \textit{\textbf{Quais sugestões você daria para melhorar a disciplina:}}

$\vartriangleright$ Nenhuma.

$\vartriangleright$ Utilizar mais eficientemente o quadro branco em explicações abstratas.

$\vartriangleright$ Apenas deixo a sugestão de passar o conteúdo com mais vigor em um tom de voz mais alto, mas isso, é claro, é um pormenor dada a excelente aula.

\newpage
\chapter{Parecer do professor responsável}

O aluno Carlos David Braga Borges desenvolveu as atividades descritas neste relatório com assiduidade e responsabilidade. Os discentes da disciplina de Inteligência Computacional também avaliaram positivamente o mestrando, conforme demonstrado na Seção 4.3. Desse modo, julgo que todos os requisitos foram cumpridos para aprovação na disciplina de Estágio em Docência.

\postextual
\newpage
% ----------------------------------------------------------

% ----------------------------------------------------------
% Referências bibliográficas
% ----------------------------------------------------------

%\bibliographystyle{abntex2-num}
%\bibliography{references}

\end{document}
